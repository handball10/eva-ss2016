\newdualentry{html} % label
  {HTML}            % abbreviation
  {Hypertext Markup Language}  % long form
  {Eine textbasierte Auszeichnungssprache zur Strukturierung digitaler Dokumente wie Texte mit Hyperlinks, Bildern und anderen Inhalten.} % description
  
\newdualentry{qrcode} % label
  {QR}            % abbreviation
  {Quick Response Code}  % long form
  {Zweidimensionaler Code, der schnell durch Maschinen gelesen werden kann.} % description
  
\newdualentry{json} % label
  {JSON}            % abbreviation
  { JavaScript Object Notation}  % long form
  {Kompaktes Datenformat in leicht verständlicher Textform, in etwa vergleichbar mit XML.} % description

\newdualentry{xml} % label
  {XML}            % abbreviation
  { Extensible Markup Language}  % long form
  {Kompaktes Datenformat, in etwa vergleichbar mit JSON.} % description
  
\newdualentry{ui} % label
  {UI}            % abbreviation
  { User Interface}  % long form
  {Siehe Graphical User Interface} % description

\newdualentry{gui} % label
  {GUI}            % abbreviation
  { Graphical User Interface}  % long form
  {Frontend-Ansicht, welche dem Benutzer eines Programms ausgeliefert wird.} % description

\newdualentry{css} % label
  {CSS}            % abbreviation
  { Cascading Style Sheet}  % long form
  {Gestaltungssprache für elektronische Dokumente und Programme wie HTML-Webseiten oder JavaFX-Anwendungen.} % description

\newdualentry{http} % label
  {HTTP}            % abbreviation
  { Hypertext Transfer Protocol}  % long form
  {Protokoll zur Übertragung von Daten über ein Rechnernetz.} % description

\newdualentry{url} % label
  {URL}            % abbreviation
  { Uniform Resource Locator}  % long form
  {Identifizierung einer Netzwerkressource, beispielsweise ein Server.} % description
  
\newdualentry{pojo} % label
  {POJO}            % abbreviation
  { Plain Old Java Object}  % long form
  {Ein Java-Objekt im herkömmlichen Sinne.} % description
  
\newacronym{token} % label
  {Token}            % abbreviation
  {Methode zur Autorisierung von Software-Diensten.} % description
  
\newdualentry{sql} % label
  {SQL}            % abbreviation
  { Structured Query Language}  % long form
  {Datenbanksprache, um durch Datenbanken zu navigieren, Bearbeitungen, Löschvorgänge und Neueintragungen vorzunehmen.} % description
  
\newdualentry{dbms} % label
  {DBMS}            % abbreviation
  { Database Management System}  % long form
  {System, welches die Datenbank aufbaut und verwaltet. Ziel ist ein möglichst effizientes und einfach zu bedienendes System.} % description

%---------- Beispiel reiner Glossareintrag ----------
%\newglossaryentry{html}{ % das wird im Befehl \gls{} im Haupttext genutzt, also \gls{html}
%	name=HTML, % das ist der Name des Eintrags im Glossar
%    text=HTML, % das wird an der Stelle des Befehls im Haupttext ausgegeben.
%    description={Hypertext Markup Language. Eine textbasierte Auszeichnungssprache zur Strukturierung digitaler Dokumente wie Texte mit Hyperlinks, Bildern und anderen Inhalten.},
%    sort=html, %hier unnötig, aber bei Wörtern mit Latex-Sonderzeichen wäre es sinnvoll, hiermit klar zu machen, wos einsortiert werden sollte
%    plural=HTML %hier schwachsinnig, könnte aber angegeben werden, um im Haupttext mit \glspl zu arbeiten
%}

%---------- Beispiel reine Abkürzung ----------
%\newacronym{Label}{Name}{Auflösung}

%---------- Beispiel sowohl als auch ----------
%\newdualentry{owd} % label
%  {OWD}            % abbreviation
%  {One-Way Delay}  % long form
%  {The time a packet uses through a network from one host to another} % description
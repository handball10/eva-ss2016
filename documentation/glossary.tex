\newdualentry{html} % label
  {HTML}            % abbreviation
  {Hypertext Markup Language}  % long form
  {Eine textbasierte Auszeichnungssprache zur Strukturierung digitaler Dokumente wie Texte mit Hyperlinks, Bildern und anderen Inhalten } % description
 
  
\newdualentry{json} % label
  {JSON}            % abbreviation
  { JavaScript Object Notation}  % long form
  {Kompaktes Datenformat in leicht verständlicher Textform, in etwa vergleichbar mit XML } % description

\newdualentry{css} % label
  {CSS}            % abbreviation
  { Cascading Style Sheet}  % long form
  {Gestaltungssprache für elektronische Dokumente und Programme wie HTML-Webseiten oder JavaFX-Anwendungen } % description


\newdualentry{url} % label
  {URL}            % abbreviation
  { Uniform Resource Locator}  % long form
  {Identifizierung einer Netzwerkressource, beispielsweise ein Server } % description
    
  
\newdualentry{nosql} % label
  {NoSQL}            % abbreviation
  { No Structured Query Language}  % long form
  {Datenbank mit nicht- relationalem Ansatz } % description
  
\newdualentry{ajax} % label
{AJAX}            % abbreviation
{Asynchronous JavaScript and XML}  % long form
{Konzept der Asynchronen Datenübertragung } % description

\newdualentry{javascript} % label
{JS}            % abbreviation
{JavaScript}  % long form
{Skriptsprache für das Web } % description



%---------- Beispiel reiner Glossareintrag ----------
%\newglossaryentry{html}{ % das wird im Befehl \gls{} im Haupttext genutzt, also \gls{html}
%	name=HTML, % das ist der Name des Eintrags im Glossar
%    text=HTML, % das wird an der Stelle des Befehls im Haupttext ausgegeben.
%    description={Hypertext Markup Language. Eine textbasierte Auszeichnungssprache zur Strukturierung digitaler Dokumente wie Texte mit Hyperlinks, Bildern und anderen Inhalten.},
%    sort=html, %hier unnötig, aber bei Wörtern mit Latex-Sonderzeichen wäre es sinnvoll, hiermit klar zu machen, wos einsortiert werden sollte
%    plural=HTML %hier schwachsinnig, könnte aber angegeben werden, um im Haupttext mit \glspl zu arbeiten
%}

%---------- Beispiel reine Abkürzung ----------
%\newacronym{Label}{Name}{Auflösung}

%---------- Beispiel sowohl als auch ----------
%\newdualentry{owd} % label
%  {OWD}            % abbreviation
%  {One-Way Delay}  % long form
%  {The time a packet uses through a network from one host to another} % description
%Packages:
\usepackage{geometry} % Ränder
\usepackage{settings/packages/bookabstract} % Abstract für ein Buch
\usepackage[ngerman]{babel} % Alles deutsch
\usepackage[utf8]{inputenc} % Umlaute
\usepackage[T1]{fontenc} % Auch für Umlaute
\usepackage[german=quotes]{csquotes} %Für Zitate, braucht bibtex
\usepackage{xcolor} % Farbdefinitionen


\usepackage{fancyhdr} % Für Header und Footer
\usepackage[]{listings} %Codeblöcke
\usepackage{accsupp} % Bessere Accessability des PDF (z.B. Code ohne Zeilennummern herauskopieren
\usepackage{xcolor} % Farben
\usepackage{blindtext} % Beispieltext
\usepackage[compact]{titlesec} % Überschriften anpassen
\usepackage[scaled]{helvet} % Helvetica laden
\usepackage{lastpage} % Gesamtseitenanzahl ermitteln
\usepackage{tikz} %Z eichnen
\usepackage{microtype} % Text sperren
\usepackage{calc} % Berechnungen bei Maßangaben
\usepackage[colorinlistoftodos,prependcaption,textsize=scriptsize]{todonotes} % todos vermerken
\usepackage{float} %B esseres Positionieren von Bildern
\usepackage[justification=centering]{caption} % Anpassen von Captions
\usepackage{subcaption} %Anpassen von Subcaptions
\usepackage{wrapfig} % Bilder umfließen
\usepackage{parskip} % Leerzeile = Abasatz
\usepackage{changepage} % Ermitteln, ob gerade/ungerade Seite
\usepackage{tocloft} % Inhaltsverzeichnis anpassen
\usepackage{xhfill} %Zeile mit Linie auffüllen
\usepackage{titling} %Titelseite bearbeiten
\usepackage{etoolbox} % Programmiertechnische Tricks
\usepackage{settings/packages/chaptersinlof} % Fügt Kapitelüberschriften ins Abbildungsverzeichnis ein
\usepackage{settings/packages/chaptersinlot} % Fügt Kapitelüberschriften ins Tabellenverzeichnis ein
\usepackage{settings/packages/chaptersinlol} % Fügt Kapitelüberschriften ins Listingverzeichnis ein
\usepackage{hyperref} % Verlinkungen innerhalb des Dokuments
\usepackage[style=alphabetic,sorting=nyt,sortcites=true,autopunct=true,autolang=hyphen,hyperref=true,abbreviate=false,backref=true,backend=biber]{biblatex} % Literaturverzeichnis
\usepackage{colortbl} % Randfarben von Tabellen
\usepackage{xargs} % Use more than one optional parameter in a new commands
\usepackage[export]{adjustbox} % Bilder umrahmen
\usepackage{xparse}

%Glossar
\usepackage{glossaries} 
\setacronymstyle{footnote-sc}
\DeclareDocumentCommand{\newdualentry}{ O{} O{} m m m m } {
  \newglossaryentry{gls-#3}{name={#5},text={#5\glsadd{#3}},
    description={#6},#1
  }
  \makeglossaries
  \newacronym[#2]{#3}{#4}{\glsadd{gls-#3}\glsseeformat{gls-#3}{}}
}
\makeglossaries
\usepackage[xindy]{imakeidx}
\makeindex
\loadglsentries[main]{glossary}

%To Dos
\newcommandx{\unsure}[2][1=]{\todo[linecolor=red,backgroundcolor=red!25,bordercolor=red,#1]{#2}}
\newcommandx{\change}[2][1=]{\todo[linecolor=blue,backgroundcolor=blue!25,bordercolor=blue,#1]{#2}}
\newcommandx{\info}[2][1=]{\todo[linecolor=OliveGreen,backgroundcolor=OliveGreen!25,bordercolor=OliveGreen,#1]{#2}}
\newcommandx{\improvement}[2][1=]{\todo[linecolor=Plum,backgroundcolor=Plum!25,bordercolor=Plum,#1]{#2}}
\newcommandx{\thiswillnotshow}[2][1=]{\todo[disable,#1]{#2}}

%Colors
\definecolor{middlegray}{rgb}{0.5,0.5,0.5}
\definecolor{orange}{rgb}{0.8,0.3,0.3}
\definecolor{lightgray}{rgb}{0.8,0.8,0.8}
\definecolor{yac}{rgb}{0.6,0.6,0.1}
\definecolor{thmgreen}{RGB}{128,186,36}

% Breite von Randnotizen
\setlength{\marginparwidth}{2.8cm}
%Tabellen
\arrayrulecolor{black}
\setlength{\arrayrulewidth}{1pt}

%includes advanced settings
%Überschriften
\titleformat{\chapter}[hang] 
{\normalfont\LARGE\bfseries\sffamily}{\thechapter}{1em}{} % Kapitelüberschrift Optik
\titleformat{\section}[hang] 
{\normalfont\Large\bfseries\sffamily}{\thesection}{1em}{} % Abschnittsüberschrift Optik
\titlespacing*{\section}{0pt}{0.5cm}{-2pt plus 0pt minus 0pt} %Abschnittsüberschrift Abstände
\titleformat{\subsection}[hang] 
{\normalfont\large\bfseries\sffamily}{\thesubsection}{1em}{} %Unterabschnittsüberschrift Optik
\titlespacing*{\subsection}{0pt}{0.5cm plus 2mm minus 2mm}{-3pt plus 0pt minus 0pt} % ... Abstände
\titleformat{\subsubsection}[hang] 
{\normalfont\large\bfseries\sffamily}{\thesubsubsection}{1em}{} %Unterunterabschnittsüberschrift Optik
\titlespacing*{\subsubsection}{0pt}{0.5cm plus 2mm minus 2mm}{-3pt plus 0pt minus 0pt} % .... Abstände
\titleformat{\paragraph}[hang] 
{\normalfont\normalsize\bfseries\sffamily}{}{1em}{} 
\titlespacing*{\paragraph}{0pt}{0.3cm plus 1mm minus 1mm}{0.1cm plus 0pt minus 0pt}

%Captions
\captionsetup{labelfont=sf, labelsep=newline}

%TOC
\renewcommand\cftchapfont{\normalsize\bfseries\sffamily} % Schriftart der Kapitelüberschriften
\renewcommand{\cftchapleader}{~\xrfill[0pt]{1pt}[thmgreen]} % Linien neben Kapitelnamen
\renewcommand{\cftsecleader}{~\cftdotfill{0}} % Linien neben Abschnittsüberschriften
\renewcommand{\cftsubsecleader}{~\cftdotfill{0}} % Linien neben Unterabschnittsüberschriften
\renewcommand{\cfttoctitlefont}{\sffamily\bfseries\LARGE} % Inhaltsverzeichnis Überschrift Optik

\renewcommand{\cftfigleader}{~\cftdotfill{0}} % Linien neben Unterabschnittsüberschriften in lof
\renewcommand{\cftloftitlefont}{\sffamily\bfseries\LARGE} %Inhaltsverzeichnis Überschrift Optik

\renewcommand{\cfttableader}{~\cftdotfill{0}} % Linien neben Unterabschnittsüberschriften in lot
\renewcommand{\cftlottitlefont}{\sffamily\bfseries\LARGE} %Inhaltsverzeichnis Überschrift Optik

%List of Listings anpassbar machen
\begingroup\makeatletter
\let\newcounter\@gobble\let\setcounter\@gobbletwo
  \globaldefs\@ne \let\c@loldepth\@ne
  \newlistof{listings}{lol}{\lstlistlistingname}
  \newlistentry{lstlisting}{lol}{0}
\endgroup
\renewcommand{\cftlstlistingleader}{~\cftdotfill{0}} %Linie neben einem Listing

%Abstände
\setlength{\parskip}{6pt} % Abstand nach Absatz
\setlength{\columnsep}{16pt} %Spaltenabstand
\setlength{\intextsep}{16pt} %Abstand zwischen Text und z.B. Bildern
\renewcommand{\baselinestretch}{1.05} % Zeilenabstand

% Strafpunkte, um Phänomene zu vermeiden
\hyphenpenalty=5000
\widowpenalty=5000
\clubpenalty=5000

%Aufzählungen
\renewcommand\labelitemi{\scriptsize$\bullet$}

\renewcommand*{\maketitle}{
\thispagestyle{empty}
\par
\vspace{5cm}
\hspace{-3.8cm}
\noindent\colorbox{thmgreen}{
\hspace{3.3cm}
\parbox{\dimexpr\paperwidth-2\fboxsep}{\vspace{10pt}\color{white}
\parbox{\textwidth}{\fontsize{22}{26}\selectfont\sffamily\bfseries\raggedright\thetitle}\hfill\vspace{5pt}
}
}
\vspace{1mm}
\par
{\sffamily\ourSubTitle}
\vspace{5cm}
\par
\hspace{0.4\textwidth}\parbox{.6\linewidth}{\fontsize{12}{14}\selectfont\raggedleft\sffamily Eine Arbeit von\\\theauthor\\\vspace{3mm}entstanden im Sommersemester 2016 im Fach \textbf{„Entwicklung verteilter Anwendung“ bei\\Prof. Dr. Karim Kremer}\\(\thedate)
}
\vspace{4cm}
\par
{
\raggedleft
\includegraphics[width=6cm]{images/THMlogo.png}\\
\vspace{0.3cm}
{\sffamily
Technische Hochschule Mittelhessen\\
Wilhelm-Leuschner-Straße 13\\
61169 Friedberg\\}
}
}
%Zeilennummern werden nicht kopiert
\newcommand{\noncopynumber}[1]{
    \BeginAccSupp{method=escape,ActualText={}}
    #1
    \EndAccSupp{}
}

%Styling
 \lstset{
   breaklines=true,
   basicstyle=\ttfamily\footnotesize,
   keywordstyle=\bfseries\ttfamily\color{orange},
   stringstyle=\color{thmgreen}\ttfamily,
   commentstyle=\color{middlegray}\ttfamily,
   emph={square}, 
   emphstyle=\color{blue}\texttt,
   emph={[2]root,base},
   emphstyle={[2]\color{yac}\texttt},
   showstringspaces=false,
   flexiblecolumns=false,
   tabsize=2,
   numbers=left,
   numberstyle=\tiny\color{gray}\noncopynumber,
   numberblanklines=false,
   stepnumber=1,
   numbersep=5pt,
   xleftmargin=10pt,
   aboveskip=5pt plus 5pt,
   belowskip=5pt plus 5pt,
   captionpos=t
 }
 
%Umlaute im Quellcode
 \lstset{literate=%
{Ö}{{\"O}}1
{Ä}{{\"A}}1
{Ü}{{\"U}}1
{ß}{{\ss}}2
{ü}{{\"u}}1
{ä}{{\"a}}1
{ö}{{\"o}}1
}

%Code Highlighting für JSON
\lstdefinelanguage{json}{
    basicstyle=\normalfont\ttfamily,
    numbers=left,
    numberstyle=\scriptsize,
    stepnumber=1,
    numbersep=8pt,
    showstringspaces=false,
    breaklines=true,
    literate=
     *{0}{{{\color{blue}0}}}{1}
      {1}{{{\color{blue}1}}}{1}
      {2}{{{\color{blue}2}}}{1}
      {3}{{{\color{blue}3}}}{1}
      {4}{{{\color{blue}4}}}{1}
      {5}{{{\color{blue}5}}}{1}
      {6}{{{\color{blue}6}}}{1}
      {7}{{{\color{blue}7}}}{1}
      {8}{{{\color{blue}8}}}{1}
      {9}{{{\color{blue}9}}}{1}
      {:}{{{\color{yac}{:}}}}{1}
      {,}{{{\color{yac}{,}}}}{1}
      {\{}{{{\color{orange}{\{}}}}{1}
      {\}}{{{\color{orange}{\}}}}}{1}
      {[}{{{\color{orange}{[}}}}{1}
      {]}{{{\color{orange}{]}}}}{1},
}

\definecolor{lightgray}{rgb}{.9,.9,.9}
\definecolor{darkgray}{rgb}{.4,.4,.4}
\definecolor{purple}{rgb}{0.65, 0.12, 0.82}

\lstdefinelanguage{JavaScript}{
	keywords={typeof, new, true, false, catch, function, return, null, catch, switch, var, if, in, while, do, else, case, break},
	keywordstyle=\color{blue}\bfseries,
	ndkeywords={class, export, boolean, throw, implements, import, this},
	ndkeywordstyle=\color{darkgray}\bfseries,
	identifierstyle=\color{black},
	sensitive=false,
	comment=[l]{//},
	morecomment=[s]{/*}{*/},
	commentstyle=\color{purple}\ttfamily,
	stringstyle=\color{red}\ttfamily,
	morestring=[b]',
	morestring=[b]"
}
\geometry{top=3.5cm,bottom=3.cm,left=3.7cm,right=3.2cm,headsep=15pt,footskip=1.7cm, a4paper, headheight=1.1cm}

\renewcommand{\headrulewidth}{0pt} % Width of the rule under the header
\renewcommand{\footrulewidth}{0pt} % Removes the rule in the footer

%Header
\pagestyle{fancy}
\fancypagestyle{fancyStd}{
\fancyhf{}
\fancyfoot[LE, RO]{\sffamily\thepage\ von \pageref{LastPage}}
\fancyfoot[LO]{\begin{minipage}{11.5cm}\scriptsize\sffamily\color{middlegray}THM SS16\end{minipage}}
\fancypagestyle{plain}{\fancyhead{}\renewcommand{\headrulewidth}{0pt}} % Style for when a plain pagestyle is specified

\fancyhead[LE]{
\hspace*{-34mm}\colorbox{thmgreen}{\makebox[\textwidth+3.2cm][l]{\scriptsize\hspace*{33mm}\color{white}\textls*[10]{ \sffamily\textbf{\leftmark}}}}
}
\fancyhead[RE]{
%\footnotesize\sffamily \ourBookDate
\begin{tikzpicture}[remember picture,overlay]
\node[yshift=-3cm, xshift=3.2cm, inner sep=0pt,outer sep=0pt, name=topleft] at (current page.north west){};
\node[yshift=-3cm, xshift=-3.7cm, inner sep=0pt,outer sep=0pt, name=topright] at (current page.north east){};
\draw[line width=1pt, black](topleft)--(topright);
\end{tikzpicture}
}

\fancyhead[RO]{
\hspace*{0mm}\colorbox{thmgreen}{\makebox[\textwidth+3.2cm][l]{\scriptsize\hspace*{0mm}\color{white}\textls*[10]{ \sffamily\textbf{\hfill\rightmark\hspace{3.3cm}}}}}
}
\fancyhead[LO]{
\begin{tikzpicture}[remember picture,overlay]
\node[yshift=-3cm, xshift=3.7cm, inner sep=0pt,outer sep=0pt, name=topleft] at (current page.north west){};
\node[yshift=-3cm, xshift=-3.2cm, inner sep=0pt,outer sep=0pt, name=topright] at (current page.north east){};
\draw[line width=1pt, black](topleft)--(topright);
\end{tikzpicture}
}}

% Removes the header from odd empty pages at the end of chapters
\makeatletter
\renewcommand{\cleardoublepage}{
\clearpage\ifodd\c@page\else
\hbox{}
\vspace*{\fill}
\thispagestyle{empty}
\newpage
\fi}

%Römische Nummerierung zu Beginn
\fancypagestyle{roman}{%
\fancyhf{} % clear all header and footer fields
\fancyfoot[RO]{\MakeUppercase\thepage}
\fancyfoot[LE]{\MakeUppercase\thepage}
} % except the center
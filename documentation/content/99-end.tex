\chapter{Fazit}
Die Entwicklung dieses Projektes hat einige sehr interessante Aspekte der interaktiven und innovativen Web-Entwicklung hervorgebracht.
Angefangen bei der Verwendung der Firebase-Platform von Google. Diese bot einen Funktionsumfang, der, im Falle einer Selbstentwicklung,
einem Entwickler einen sehr hohen Aufwand erspart. Zu Beginn ist Firebase kostenlos und ermöglicht auch Studenten, die Platform ohne Kosten zu nutzen.\\
In jedem Fall konnte sich die Plattform für die weitere Nutzung in Uni-Projekten und darüber hinaus empfehlen.
\\
Der entstandene Prototyp funktioniert sehr gut. Vor allem die Synchronisation der Daten zwischen verschiedenen Clients läuft erstaunlich schnell.
Eine Weiterentwicklung des Prototyps ist voraussichtlich nicht vorgesehen
\\
Zusammengefasst lässt sich sagen, dass das Projekt uns persönlich in unseren Interessensgebieten durchaus weitergebracht hat und neue Technologien dadurch zu unserem
Repertoire gehören.\\
Kommende Projekte würden wir wieder mit dieser Technologie-Kombination entwickeln.
\subsection{Technische Umsetzung der Datenbank-Anbindung}
Um die Arbeit mit Datensätzen und Daten-Listen zu erleichtern, wurde ein System nach einem Model-Collection-Pattern entwickelt.
Demnach sind einzelne Datensätze über Models organisiert. Diese Models werden in typisierten Collections zusammengefasst.
Im nachfolgenden werden der Aufbau und die Funktionsweise beider Konstrukte erläutert.

\subsubsection{Models}

Als Model bezeichnet man eine Klasse, die die Struktur eines Objektes beschreibt. Ein Model hat einen Constructur, über welchen sich ein neues Objekt dieses Types generieren lässt.
Im Fall der vorliegenden Anwendung bilden drei verschiedene Models die Datenstruktur des Systems ab.\\
Dazu gehören:
\begin{description}
\item[Booking]\hfill \\
Hält eine Buchung
\item[Customer]\hfill \\
Beschreibt den Kunden
\item[Resource]\hfill \\
Beschreibt das vermietete Objekt
\end{description}

Über ein Model lässt sich ein Objekt eines spezifischen Types generieren.
Innerhalb der AngularJS-Anwendung wird das Model als Factory-Service implementiert. Dies ermöglicht es, beliebig viele Instanzen dieses Models zu generieren.
Im nachfolgenden der Aufbau des Resource-Models:

\begin{lstlisting}[language=Javascript, label=code_ResourceModel, caption=Hauptteil des Reosurce-Models]
angular.module('bookingCalendarApp')
    .factory('Resource', function ($log) {
        function Resource(properties){

            var self = this;

            this.Id    = undefined;
            this.Size  = undefined;
            this.Name  = undefined;

            function extend(properties){
                [...]
            }

            extend(properties);

        }

        return Resource;
    });
\end{lstlisting}

Wesentlicher Bestandteil des Models sind seine Properties, welche die beschreibbaren Felder darstellen. Bei jeder Instanziierung wird die Methode \texttt{extend\(\)} aufgerufen, welche die
übergebenen Werte auf das Model überträgt.
Innerhalb der Anwendung können Models in allen möglichen Komponenten verwendet werden. Zu beachten ist, dass es anders als bei komplett objektorientierten Sprachen wie bspw. Java,
keine Getter- oder Setter-Methoden gibt und die Properties des Models alle als Public angesehen werden können.


\subsubsection{Remote Objects}
Der Service \texttt{RemoteObjects} agiert als Factory zum Erstellen einer Collection für spezifische Models. Sie ist über einen Angular-Service als Singleton-Klasse implementiert und besitzt
 nur eine Mthode zum Generieren einer neuen Collection.

 \begin{lstlisting}[language=Javascript, label=code_RemoteObject, caption=Code des RemoteObjects-Service]
 angular.module('bookingCalendarApp')
     .service('RemoteObject', function (Collection) {
         var service = {};
         service.createCollection = function(name, path, Model, realtime){

             if(!name || !path || !Model){
                 throw new Error('remoteObject:: Missing parameter in Object')
             }
              return new Collection({
                 name : name,
                 path : path,
                 realtime : realtime,
                 Model : Model
             });
         };
         return service;
     });
\end{lstlisting}

Wie zu sehen, fordert die Factory bei der Instanziierung bis zu vier Parameter. Über den Parameter \texttt{name} wird der Name der Collection bestimmt, \texttt{path} ist der Pfad, unter welchem die Objekte dieses Types
innerhalb von Firebase gefunden werden können, \textt{Model} ist das Model-Objekt und \texttt{realtime} ein Indikator, ob die Collection auf Echtzeit-Updates der Datenbank hören soll.

Rückgabewert der Methode \texttt{createCollection()} ist immer die Collection des jeweiligen Model-Types.
Wie diese Collections funktionieren, wird im nächsten Kapitel erläutert.

\subsection{Collections}

Eine Collection bezeichnet eine Art Sammlung für einen bestimmten Datentypen. Sie stellt Funktionen zur Verfügung, mit welchen der Entwickler einfach Datenbank-Abfragen generieren und Daten speichern kann.
Ebenso wie der RemoteObjects-Service, ist auch eine Collection als Service imlementiert, sodass nur jeweils eine für jedes Model existieren kann.

\subsubsection{Instanziierung}

\subsubsection{Methoden}

\subsubsection[Echtzeit-Anbindung]
\chapter{Einführung}
Ich bin der Einführungstext
\blindtext[2]\cite{BB2016}
\section{Test1}
Ich bin ein Textabschnitt
\blindtext[1]
\subsection{Ebene runter}
\blindtext[2]
\begin{figure}[H]
\centering
\includegraphics[width=0.5\textwidth]{images/AbstractFactory.pdf}
\caption{test}
\label{testfigure}
\end{figure}
\subsubsection{Noch eine Ebene}
\blindtext
\blindtext[1]\ref{fig:figlabel}

\paragraph{Ein Absatz}
\blindtext
\begin{wrapfigure}{o}{0.5\textwidth} 
    \centering
    \includegraphics[width=0.5\textwidth]{images/AbstractFactory.pdf}
    \caption{Bildunterschrift und}
    \label{fig:figlabel}
\end{wrapfigure}
\blindtext\footnote{Das sind wichtige Informationen}
\begin{lstlisting}[language=java, caption=Das ist hochkomplexer Code]
public class HelloWorld 
{
 
       public static void main (String[] args)
       {
             // Ausgabe Hello World!
             System.out.println("Hello World!");
       }
}
\end{lstlisting}
\blindtext[4]
\begin{wrapfigure}{o}{0.5\textwidth}
\begin{lstlisting}[language=java, caption=Das ist hochkomplexer Code, belowskip=0pt]
public class HelloWorld 
{
 
       public static void main (String[] args)
       {
             // Ausgabe Hello World!
             System.out.println("Hello World!");
       }
}
\end{lstlisting}
\end{wrapfigure}
\blindtext
\begin{table}[h]
\centering
\caption{My caption}
\label{my-label}
\begin{tabular}{|l|l|l|l|l|}
\hline
fgfhg &     &     &  &  \\ \hline
      &     & gfg &  &  \\ \hline
      & gfh &     &  &  \\ \hline
fgh   &     &     &  &  \\ \hline
\end{tabular}
\end{table}
\blindtext

\begin{wraptable}{o}{0.5\textwidth}
\caption{My caption}
\label{my-new-label}
\begin{tabular}{|l|l|l|l|l|}
\hline
fgfhg &     &     &  &  \\ \hline
      &     & gfg &  &  \\ \hline
      & gfh &     &  &  \\ \hline
fgh   &     &     &  &  \\ \hline
\end{tabular}
\end{wraptable}
\blindtext[2]
\begin{itemize}
\item ich bin ein Listenpunkt
\item Ich auch
\end{itemize}